\documentclass{beamer}

%Sample should be placed in parent directory of the template directory!
%This document needs to be compiled TWICE on first run
%This file needs to be compiled with XeLaTeX due to its reliance upon the fontspec package.

\title{Quantum Geleiding}
\subtitle{Presenteren \& Communiceren}
\date{27 Mei 2015}
\author[Snijders]{Max Snijders \\ \texttt{msnijders@physics.leidenuniv.nl}}

%\setbeameroption{show notes}
%\setbeamertemplate{note page}[plain]

\usepackage[LastSlideNotCounted,ProgressBar,NoPageCounter]{beamer-template-mk-i/beamerthememxmki}
\hypersetup{pdfpagemode=FullScreen}

%\usepackage{pgfpages}
%\pgfpagesuselayout{4 on 1}[a4paper]

\newcommand*{\openquote}{\tikz[remember picture,overlay,xshift=-15pt,yshift=-10pt]
	\node (OQ) {\fontsize{60}{60}\selectfont``};\kern0pt}
\newcommand*{\closequote}{\tikz[remember picture,overlay,xshift=15pt,yshift=10pt]
	\node (CQ) {\fontsize{60}{60}\selectfont''};}
% select a colour for the shading
\definecolor{shadecolor}{named}{white}
% wrap everything in its own environment
\newenvironment{shadequote}%
{\begin{quote}\openquote}
		{\hfill\closequote\end{quote}}

%Presentatie mag maar 8 minuten duren!

\begin{document}
	
	%Eerste slide, deze is voor bij het voorstellen e.d.
	\section{Presenteren en Communiceren}
	\begin{frame}
		\titlepage
	\end{frame}
	
	\begin{frame}{Wat is geleiding?}
				\begin{itemize}
					\item<2-> Macroscopisch \onslide<4->{$\rightarrow$ Klassiek}
					\item<3-> Microscopisch \onslide<5->{$\rightarrow$ Quantum}
				\end{itemize}
				\note{Deze slide maakt het onderscheid tussen geleiding op grote en kleine schaal duidelijk}
	\end{frame}
	
	\begin{frame}{Wat is geleiding?}
		\begin{shadequote}
			Geleiding op atomaire schaal is de kans voor een electron om door een structuur heen te komen
		\end{shadequote}
		\note{Opzet van deze slide is om het concept van geleiding als een kans (transmissiecoefficient) duidelijk te maken.}
	\end{frame}
	
	\begin{frame}{De transmissieco\"efficient}
		Hier een simpel model voor de transmissiecoefficient
		\note{Hier toon ik graag een quantum systeem waarbij er duidelijk sprake is van een transmissiecoefficient, zoals een simpele potentiaalbariere.}
	\end{frame}
	
	\begin{frame}{Theoretisch Model}
		\note{Deze slide moet een simpel theoretisch model duidelijk maken waaruit we de transmissiecoefficient kunnen gaan bepalen}
	\end{frame}
	
	\begin{frame}{$G_0$}
		Voorspelling uit model
		\note{Op deze slide staat groot de numerieke waarde (en de formule ervoor) van $G_0$.}
	\end{frame}
	
	\begin{frame}{Opstelling}
		Simpel diagrammetje
		\note{Simpel en strak diagrammetje met daarop de opstelling (breken en maken van atomair contact)}
	\end{frame}
	
	\begin{frame}{Geleidingstrappetje}
		Hier een plaatje van een geleidingstrappetje
		\note{Plotje van een geleingstrappetje waarbij de kwantisatie van geleiding duidelijk zichtbaar is.}
	\end{frame}
	
	\begin{frame}{Histogram}
		Hier een geleidingshistogram
		\note{Histogram van de geleiding waarbij de kwantisatie duidelijk zichtbaar is.}
	\end{frame}
	
	%\section{Einde}
	\setbeamercolor{background canvas}{bg=black}
	\begin{frame}[plain]\end{frame}
	
\end{document}